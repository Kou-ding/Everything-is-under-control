\documentclass[hidelinks,titlepage, 12pt, a4paper]{article}

%%%%%%%%%%% INFO %%%%%%%%%%%
% The "preamble"  document contains all the packages used in this project and their particular settings.
%It basically is a TEMPLATE for the documents of this project.
%%%%%%%%%%%%%%%%%%%%%%%%%%%




% %The "polyglossia" package allows the use of Greek fonts alongside English ones
% \usepackage{polyglossia}
%     \newfontfamily\greekfont[Script=Greek]{Linux Libertine O}
%     \newfontfamily\greekfontsf[Script=Greek]{Linux Libertine O}
%     \setdefaultlanguage{greek}
%     \setotherlanguage{english}

\usepackage{polyglossia}
\setdefaultlanguage{greek}
\setotherlanguage{english}

\usepackage{fontspec}
\setmainfont[
    Script=Greek,  % Ensures proper Greek rendering
    Ligatures=TeX, % Standard TeX ligatures (like `` and '')
]{Courier New}     % Main font for Greek and Latin text

% The "amsmath" package allows the inclusion of mathematica symbols in the LaTeX document
\usepackage{amsmath}
\usepackage{caption}
\usepackage{subcaption}
% The "array" package allows the center allignment of columns with wrapped content
\usepackage{array}
    \newcolumntype{P}[1]{>{\centering\arraybackslash}p{#1}}

%  The "ulem" package allows strikethrough effect in text
\usepackage{ulem}

% The "graphicx" package allows insertion of images
\usepackage{graphicx}

% The "subfig" packages allows the use of figures that consist of more than one images
\usepackage{subfig}

% The "wrapfig" package allows the insertion of images on the side of text and in other custom positions
\usepackage{wrapfig}

% The "ragged2e" package allows justified text (pliri stixoisi)
\usepackage[document]{ragged2e}

% The "xcolor" package allows colouring tables
\usepackage[table,xcdraw]{xcolor}

% The "hyperref" package allows adding links to LaTeX document
\usepackage{hyperref}

%  The "todonotes" package allows adding nice inline and side notes 
\usepackage[colorinlistoftodos]{todonotes}\setlength{\marginparwidth}{2.5cm}

% The "tcolorbox" package allows creating coloured boxes of text in the main body of text.
\usepackage[most]{tcolorbox}
        \newtcolorbox {theorem} [2] []{colback=green!5,colframe=green!35!black,colbacktitle=green!35!black,fonttitle=\bfseries,enhanced,attach boxed title to top left={yshift=-2mm,xshift=-7mm},width=.9\textwidth,arc=.7mm, title={#2},#1}%{th}
        
        \newtcolorbox {defn} [2] []{colback=blue!5,colframe=cyan!35!black,colbacktitle=blue!35!black,fonttitle=\bfseries,enhanced, attach boxed title to top left= {yshift=-2mm,xshift=-2mm}, title={#2},#1}%{def}

        \newtcolorbox {note} [2] []{colback=yellow!10!white,colframe=red!75!black,colbacktitle=red!75!black,fonttitle=\bfseries,enhanced, attach boxed title to top left= {yshift=-2mm,xshift=-2mm}, title={#2},#1}%{note}




%%%%%%%%%%%%%%%%%%%%%%%%%%%%%%%%%%%%%%%%%%
%%%%%%%%%% Page layout settings %%%%%%%%%%
% The "geometry" package allows the formating of the writable space in a page
\usepackage{geometry}
        \geometry{
            a4paper,
            total={150mm,247mm},
            left=30mm,
            top=30mm
        }

% The "parindent"  package allows setting the paragraph indent of the document (i.e. the empty space at the beggining of each paragraph)
\parindent=20pt
% By default, TeX indents the first line of each paragraphs by 20pt
\usepackage{parskip}
\parskip=3.5pt

% The "relsize"  package allows using font sizes larger than 'Huge'  and smaller than 'tiny'
\usepackage{relsize}
        %The "\renewcommand\RSlargest" command sets the maximum font size allowed by the 'relsize '  package
        \renewcommand\RSlargest{75pt}

%%%%%%%%%%  End - Page layout settings %%%%%%%%%% 
%%%%%%%%%%%%%%%%%%%%%%%%%%%%%%%%%%%%%%%%%%%%%%%%%

% The "float" package allows the placement of figures, images etc. in specific places in between text
\usepackage{float}

% The "afterpage" package allows the insertion of blank pages (eg. a blank page after the 'titlepage')
\usepackage{afterpage}
        \newcommand\blankpage{%
            \null
            \thispagestyle{empty}%
            %\addtocounter{page}{-1}%
            \newpage}


% The "mathtools"  package allows  the insertion of text, formulas etc. above arrows (=> etc.) in mathematical expressions
\usepackage{mathtools}

% The "mathrsfs" package allows the use of curly math letters, like the Lagrange 'L'
\usepackage{ mathrsfs }

% The "cancel" package allows diagonal arrow-kind-strikethrough in math expressions
        %\usepackage{cancel}
\usepackage[makeroom]{cancel}
    \makeatletter
    % #1, #2 offset of label   #6 extra width to clear arrowhead
    % #3, #4 vector direction  #7 superscript label style
    % #5 vector width          #8 superscript label
    \def\cantox@vector#1#2#3#4#5#6#7#8{%
        \dimen@.5\p@
        \setbox\z@\vbox{\boxmaxdepth.5\p@
        \hbox{\kern-1.2\p@\kern#1\dimen@$#7{#8}\m@th$}}%
    \ifx\canto@fil\hidewidth  \wd\z@\z@ \else \kern-#6\unitlength \fi
        \ooalign{%
        \canto@fil$\m@th \CancelColor
        \vcenter{\hbox{\dimen@#6\unitlength \kern\dimen@
            \multiply\dimen@#4\divide\dimen@#3 \vrule\@depth\dimen@\@width\z@
            \vector(#3,-#4){#5}%
    }}_{\raise-#2\dimen@\copy\z@\kern-\scriptspace}$%
        \canto@fil \cr
        \hfil \box\@tempboxa \kern\wd\z@ \hfil \cr}}
    \def\bcancelto#1#2{\let\canto@vector\cantox@vector\cancelto{#1}{#2}}
    \makeatother

% The "tabto" package allows the use of the regular 'Tab' funcionality of inserting empty space
\include{tabto}


% The "amssymb" package allows the use of yet more mathematical symbols
\usepackage{amssymb}

%The "enumitem" package allows the use of lists numbered alphabetically etc.
\usepackage{enumitem}

%The "diagbox", "color", "vcell", "adjustbox" packages allow custom adjustments to tables
\usepackage{diagbox}
\usepackage{color}
\usepackage{vcell}
\usepackage{adjustbox}

%The "changepage" package allows the ????????
\usepackage{changepage} 

%The "unicode-math" package allows the use of more Greek symbols in equations
\usepackage{unicode-math}

%tables
\usepackage{multirow}
\usepackage[table,xcdraw]{xcolor}
\usepackage{colortbl}

%The following two commands allow the numbering of \paragraphs and their inclusion in the Table of Contents.
% further explanation: https://tex.stackexchange.com/questions/186981/is-there-a-subsubsubsection-command
\setcounter{tocdepth}{4}
\setcounter{secnumdepth}{4}

%The "gensymb" package allows the insertion of the 'degree' symbol
\usepackage{gensymb}

%The "xfrac" package allows the insertion of 'diagonal' fraction 
\usepackage{xfrac}

%The "multicol" package allows the insertion of lists with more than one columns
\usepackage{multicol}




\begin{document}

\begin{titlepage}
\begin{center}
    \vspace{1cm}

    \begin{figure}[h]
        \centering
        \includegraphics[width=8cm]{Images/AUTH_Banner.png}
    \end{figure}

    \vspace{2.5cm}               
    \Huge
    
    \hspace{-1cm}\textlarger[4]{ \textbf{Τ}}\textlarger[1.5]{ΗΛΕΟΠΤΙΚΑ }\textlarger[4]{\textbf{Σ}}\textlarger[1.5]{ΥΣΤΗΜΑΤΑ}  
    
            
    \vspace{2.5cm}
    \Huge
    \textbf{"ΟΛΑ ΕΙΝΑΙ ΥΠΟ ΕΛΕΓΧΟ"}
    
    \LARGE
    \vspace{2cm}
    \textbf{1ο Παραδοτέο}\\
    - Σενάριο -
            
    \vspace{2cm}
    \Large
    \vspace{0.2cm}
    Καρατζάς Αθανάσιος, ΑΕΜ 10514\\
    \vspace{0.2cm}
    Παναπακίδης Δημήτριος, ΑΕΜ 9298\\
    \vspace{0.2cm}
    Παπαδάκης Κωνσταντίνος Φώτιος, ΑΕΜ 10371\\
    \vspace{0.2cm}
    Παρασκευοπούλου Άννα, ΑΕΜ 10238
            
    \vfill
    \Large       
            
    \large
    Τμήμα Ηλεκτρολόγων Μηχανικών και Μηχανικών Υπολογιστών ΑΠΘ\\
    Σάββατο 8 Μαρτίου 2025
            
\end{center}
\date{Εαρινό Εξάμηνο 2024-2025}

%\afterpage{\blankpage} % Inserting a blank page after titlepage

\end{titlepage}

\justifying
\setcounter{page}{3} % Because this is actually page 3
\tableofcontents
%\afterpage{\blankpage} % Inserting a blank page

\newpage
\section{Sequence 1}

    \vspace{0.2cm}
    \begin{defn}[width=.95\textwidth]{Sequence Information}{}

        \vspace{0.1cm}
        \noindent $\blacktriangleright$ \textbf{ΘΕΜΑ Sequence:} Intro, Παρουσίαση θέματος
        
        \vspace{0.1cm}
        \noindent $\blacktriangleright$ \textbf{Scene 2:} Απαιτείται δημιουργία του Breaking News content πριν τα γυρίσματα
    \end{defn}



    \vspace{0.5cm}
    \subsection*{ \uline{\uline{\textcolor{gray}{ \textbf{Scene 0} [Έξω για μπύρες] [Optional] }}} } 
    \addcontentsline{toc}{section}{\protect\numberline{}1.1 \hspace{0.1cm} Scene 0 [Έξω για μπύρες]} \todo[color=green!40]{Optional scene. Maybe at a house party}

    Η φοιτήτρια είναι έξω σε μαγαζί με παρέα. Καταναλώνουν διάφορα ποτά καθώς συζητούν ανά μεταξύ τους. Γελούν, περνάν καλά.


    
    \vspace{0.5cm}
    \subsection*{ \uline{\uline{\textbf{Scene 1} [Δρόμος της επιστροφής]}} } 
    \addcontentsline{toc}{section}{\protect\numberline{}1.2 \hspace{0.1cm} Scene 1 [Δρόμος της επιστροφής]}
    
    Η φοιτήτρια βρίσκεται στον δρόμο της επιστροφής από το φοιτητικό πάρτι. ( Το σπίτι βρίσκεται πολύ κοντά στα πανεπιστήμια) Έχει πιει και για αυτό η κάμερα είναι out of focus καθ' όλη τη διάρκεια της λήψης έξω από το σπίτι της στο δρόμο της επιστροφής.


    
    \vspace{0.5cm}
    \subsection*{ \uline{\uline{\textbf{Scene 2} [Σπίτι]}} }
    \addcontentsline{toc}{section}{\protect\numberline{}1.3 \hspace{0.1cm} Scene 2 [Σπίτι]}
    
    Η κάμερα δυσκολεύεται να εστιάσει (ακόμα out of focus γιατί η φοιτήτρια είναι πιωμένη) αλλά προλαβαίνει να κάνει μια γρήγορη εστίαση έτσι ώστε η κοπέλα να μπορέσει να βάλει το κλειδί στην κλειδαριά και να μπει μέσα στο σπίτι. Όταν μπαίνει μέσα ανοίγει την τηλεόραση όπως κάνει κάθε βράδυ για να την πάρει ο ύπνος αλλά είναι τόσο κουρασμένη που δεν προλαβαίνει να ακούσει, ούτε καν να νανουριστεί από την τηλεόραση. Καθώς μπαίνει όλο και περισσότερο σε μια ενδιάμεση κατάσταση μεταξύ ύπνου και ξύπνιου την παίρνει τελικά ο ύπνος και μαζί χάνεται και κάθε ήχος εντός του δωματίου. Αφού κλείσει τα μάτια της προλαβαίνουμε να δούμε στην τηλεόραση να παίζουν Breaking News. Ο ήχος δεν είναι καθαρός και το μόνο που μισο-φαίνεται είναι ο τίτλος των νέων αυτών.

\newpage
\section{Sequence 2}

    \vspace{0.2cm}
    \begin{defn}[width=.95\textwidth]{Sequence Information}{}

        \vspace{0.1cm}
        \noindent $\blacktriangleright$ \textbf{ΘΕΜΑ Sequence:} Κυρία δράση της ιστορίας
        
        \vspace{0.1cm}
        \noindent $\blacktriangleright$ \textbf{Scene 8:} Απαιτείται δημιουργία του AI Mobile Notification content πριν τα γυρίσματα

        \vspace{0.1cm}
        \noindent $\blacktriangleright$ \textbf{Scene 9:}
        Απαιτείται δημιουργία του Breaking News content πριν τα γυρίσματα
    \end{defn}

    \vspace{0.5cm}
    \subsection*{ \uline{\uline{\textbf{Scene 1} [Πρωινό ξύπνημα]}} } 
    \addcontentsline{toc}{section}{\protect\numberline{}2.1 \hspace{0.1cm} Scene 1 [Πρωινό ξύπνημα]}

    Το ξυπνητήρι φέρνει το επόμενο πρωί όπου η τηλεόραση έχει κλείσει λόγω του χρονοδιακόπτη και η πρωταγωνίστρια ξεκινά μισοκοιμισμένη να ετοιμάζεται για να πάει στη σχολή. Σηκώνεται, βάζει τις παντόφλες της, πλένει τα δόντια της, μεταβαίνει ντυμένη μέχρι την κουζίνα, παίρνει ένα σνακ και βγαίνει από την εξώπορτα.

    \vspace{0.5cm}
    \subsection*{ \uline{\uline{\textbf{Scene 2} [Στο δρόμο για  τη σχολή]}} } 
    \addcontentsline{toc}{section}{\protect\numberline{}2.2 \hspace{0.1cm} Scene 2 [Στο δρόμο για τη σχολή]}

    Έπονται πλάνα από τον δρόμο προς το πανεπιστήμιο όπου, φορώντας ακουστικά στα αυτιά, παρατηρεί κόσμο να είναι πολύ βιαστικός και ανήσυχος. Το περπάτημα της όμως είναι γρήγορο και η ίδια δεν δίνει πολύ σημασία σε κανέναν. Φτάνει στο μετρό το οποίο παίρνει μέχρι τον σταθμό "Πανεπιστήμιο". Η μουσική που σταδιακά κλιμακώνει το μυστήριο.

    
    
    \vspace{0.5cm}
    \subsection*{ \uline{\uline{\textbf{Scene 3} [Άφιξη στη σχολή]}} } 
    \addcontentsline{toc}{section}{\protect\numberline{}2.3 \hspace{0.1cm} Scene 3 [Άφιξη στη σχολή]}

    Όταν φτάνει στο πανεπιστήμιο βλέπει πολύ λιγότερα άτομα από όσα είχαν έρθει την πρώτη μέρα στη σχολή. Τα πρόσωπα που την περιτριγυρίζουν είναι στο κινητό τους και αρκετά ανήσυχα.

    \vspace{0.5cm}
    \subsection*{  \uline{\uline{\textbf{Scene 4} [Πρώτη αλληλεπίδραση]}}  } 
    \addcontentsline{toc}{section}{\protect\numberline{}2.4 \hspace{0.1cm} Scene 4 [Πρώτη αλληλεπίδραση]}

    Ένας τύπος την πλησιάζει και της λέει:

    \begin{theorem}[width=1\textwidth]{Διάλογος}{}
        
        \textbf{[A]} \hspace{0.2cm}  Hey! Μήπως θα μπορούσα να χρησιμοποιήσω το κινητό σου. Το δικό μου δεν συνδέεται στο ίντερνετ και δεν έχω μονάδες.
        
        \textbf{[Γ]} \hspace{0.2cm}  Ναι, κανένα θέμα
        
        \textbf{[Α]} \hspace{0.2cm}  Ευχαριστώ πολύ

    \end{theorem}


   \newpage
    Ο τύπος πληκτρολογεί έναν αριθμό και βάζει το κινητό στο αυτί του

    
    \begin{theorem}[width=1\textwidth]{Διάλογος}{}
        
        \textbf{[Α]} \hspace{0.2cm}  (στο τηλέφωνο) Έλα σε παίρνω από μια συμφοιτήτρια, πως είναι ο μπαμπάς?
        
        \textbf{[Μ]} \hspace{0.2cm}  (Μητέρα αχνά) Δεν δουλεύει η πιστωτική κάρτα... Οι γιατροί αρνούνται να τον χειρουργήσουν χωρίς να πληρώσουμε όλο το ποσό. Προσπαθώ όλο το πρωί σε όλα τα ATM και κανένα δεν λειτουργεί, δεν ξέρω τι να κάνω... Έλα σε παρακαλώ να με βοηθήσεις. Ίσως κάνω κάτι λάθος?... (συναισθηματικό ύφος, αμήχανο γέλιο)

        \textbf{[Α]} \hspace{0.2cm}  (μετά από 3 δευτερόλεπτα σιγή) Που είσαι για να σε βρω?
        
        \textbf{[Μ]} \hspace{0.2cm}  Στο νοσοκομείο Παναγία.
        
        \textbf{[Α]} \hspace{0.2cm}  Εντάξει παίρνω ταξί και έρχομαι.
        
    \end{theorem}   
        


    \vspace{0.2cm}
    Έπειτα ο τύπος έχει ένα κενό βλέμμα για 5 δευτερόλεπτα και γυρνάει προς την κοπέλα με αργό βάδην λέγοντας:
    \vspace{0.2cm}

    \begin{theorem}[width=1\textwidth]{Διάλογος}{}
        
        \textbf{[Α]} \hspace{0.2cm}  Ευχαριστώ πολύ (με ένα ψεύτικο χαμόγελο)

    \end{theorem}
    



    \vspace{0.5cm}
    \subsection*{ \uline{\uline{\textbf{Scene 5} [Δεύτερη αλληλεπίδραση]}} } 
    \addcontentsline{toc}{section}{\protect\numberline{}2.5 \hspace{0.1cm} Scene 5 [Δεύτερη αλληλεπίδραση]}

    Η προσοχή της κοπέλας στρέφεται για λίγα δευτερόλεπτα μερικά θρανία πιο μπροστά σε μερικούς συμφοιτητές. Η εναλλαγή της κάμερας είναι γρήγορη, ελαφρώς τρεμάμενη και θολή στις άκρες σαν να προσομοιάζει τις γρήγορες ματιές που ρίχνει και η πρωταγωνίστρια.
    
    
    \begin{theorem}[width=1\textwidth]{Διάλογος}{}

        \textbf{[Τ1]} \hspace{0.2cm}  Προσπαθώ να μπω στο sis να δω αν βγήκε βαθμολογία αλλά όταν κάνω login η σελίδα είναι όλη λευκή. Ξέρεις τι παίζει?
    
        \textbf{[Τ2]} \hspace{0.2cm}  Δεν έχω ιδέα κάτι θα κάνουμε πάλι στο ΚΗΔ. Τους ξέρεις πως είναι χαχα, είμαστε οι beta testers τους.
    
        \textbf{[Τ1]} \hspace{0.2cm}  Χάρη μου κάνουν. Παρατείνουν την ευδαιμονία της άγνοιας.
        
    \end{theorem}
    



    \vspace{0.5cm}
    \subsection*{ \uline{\uline{\textbf{Scene 6} [Τρίτη αλληλεπίδραση]}} } 
    \addcontentsline{toc}{section}{\protect\numberline{}2.6 \hspace{0.1cm} Scene 6 [Τρίτη αλληλεπίδραση]}
    
    Εναλλαγή στο επόμενο άτομο. Πάει να συνδεθεί σε ένα παιχνίδι στο κινητό του και οι κωδικοί του δεν είναι σωστοί.

    \vspace{0.5cm}
    \subsection*{ \uline{\uline{\textbf{Scene 7} [Τέταρτη αλληλεπίδραση]}} } 
    \addcontentsline{toc}{section}{\protect\numberline{}2.7 \hspace{0.1cm} Scene 7 [Τέταρτη αλληλεπίδραση]}

    Εναλλαγή στο επόμενο άτομο (Άρης). Κάθεται στην άκρη της αίθουσας και διαβάζει ένα βιβλίο.
    
    \newpage
    \subsection*{ \uline{\uline{\textbf{Scene 8} [Ειδοποίηση]}} } 
    \addcontentsline{toc}{section}{\protect\numberline{}2.8 \hspace{0.1cm} Scene 8 [Ειδοποίηση]}
    

    Εναλλαγή στο επόμενο άτομο. Το κινητό της χτυπάει σαν τα μηνύματα της πολιτικής προστασίας με έναν δυνατό θόρυβο όμοιο με αυτό της πολιτικής προστασίας. Το ίδιο γίνεται και σε όλα τα υπόλοιπα άτομα της αίθουσας. Όλοι οι ήχοι από τα κινητά των συμμαθητών μειώνονται από την δική της αντίληψη και γίνονται θολοί. Η κοπέλα εστιάζει στην οθόνη του κινητού.


    \begin{theorem}[width=1\textwidth]{Mobile AI announcement}{}
        
        \textbf{[Κ]} \hspace{0.2cm}  (Κινητό AI voice) Από τις πρόσφατες συνομιλίες σου αποδεικνύεται η αντιγραφή σου στο μάθημα Πεδίο 2. Αυτό επηρεάζει τον κοινωνικό σου σκορ. (παύση ενός δευτερολέπτου)

    \end{theorem}
    


    Το σκορ πέφτει από 3.2 σε 2.7. Και γίνεται ένα γρήγορο flashback στο όταν αντέγραψε. Γίνεται χρήση ενός faded σχεδον ασπρομαυρου φιλτρου για να υποδηλώσουμε ότι έγινε στο παρελθον.


    \begin{theorem}[width=1\textwidth]{Mobile AI announcement}{}
        
        \textbf{[Κ]} \hspace{0.2cm}  Προσοχή! Αν το σκορ σου πέσει κάτω από 2.5 θα υπάρξουν αυστηρές κοινωνικοοικονομικές κυρώσεις! Ελπίζουμε να γίνεις και εσύ κομμάτι ενός δίκαιου και ασφαλούς μέλλοντος! :)

    \end{theorem}
    
    Όλοι οι ήχοι πλέον από θολοί γίνονται ένα βουητό καθώς το βλέμμα της γίνεται όλο και πιο έντονο. 

    


     \vspace{0.5cm}
    \subsection*{ \uline{\uline{\textbf{Scene 9} [Breaking News στο λάπτοπ ενός φοιτητή]}} }
    \addcontentsline{toc}{section}{\protect\numberline{}2.9 \hspace{0.1cm} Scene 9 [Breaking News]}

    Η πρωταγωνίστρια σαστισμένη αναζητά απαντήσεις στο περιβάλλον της και παρατηρεί πως όσο ήταν απορροφημένη στο κινητό της οι υπόλοιποι φοιτητές είχαν ήδη μαζευτεί μπροστά από μια οθόνη λάπτοπ. Καθώς πλησιάζει με διστακτικό βήμα, σιγά-σιγά, ακούγεται μια reporter να ανακοινώνει τα νέα.
    
    \begin{theorem}[width=1\textwidth]{Λόγια παρουσιάστριας}{}
        
        \textbf{[Ν]} \hspace{0.2cm}  (news reporter) Η Βαϊρέλια (μία πλασματική χώρα) έχει εξαπολύσει μια παγκόσμιας κλίμακας κυβερνοεπίθεση χάρη στη πρόσφατη ανάπτυξη ενός κβαντικού υπολογιστή ικανού να σπάσει την κρυπτογράφηση δημοσίου κλειδιού RSA. Ακόμη δεν ξέρουμε το πλήρες εύρος της επίθεσης και μένουμε διαρκώς σε επαφή με ειδικούς ενημερώνοντας σας συνεχώς για την κατάσταση των  πληροφοριακών συστημάτων εντός της χώρας καθώς και σε παγκόσμιο επίπεδο. Προς το παρόν έχουν επιβεβαιωθεί μαζικές αποκρυπτογραφήσεις δεδομένων, hacking υποδομών συνδεδεμένες με τον διαδίκτυο, μαζικές κρυπτογραφήσεις και διαγραφές δεδομένων. Ο στόχος αυτών δεν είναι πλήρως γνωστός μέχρι στιγμής ενώ δεν έχει διερευνηθεί ακόμα ποια συστήματα έχουν επηρεαστεί με τι τρόπο. Το ίντερνετ δεν είναι ασφαλές Βεβαιωθείτε ότι ενημερώνεστε από αναλογικές μορφές ενημέρ-
    \end{theorem}
    
    Η εικόνα παγώνει για λίγο εμφανίζοντας στην οθόνη glitches, RGB noise και η χροιά της φωνής αλλάζει ανεπαίσθητα. Η ρεπόρτερ πλέον χαμογελάει.
    
    \begin{theorem}[width=1\textwidth]{Λόγια παρουσιάστριας}{}
        
        \textbf{[Ν]} \hspace{0.2cm}  (news reporter) Όλα είναι υπό έλεγχο. Δεν υπάρχει κανένας λόγος ανησυχίας.
        
    \end{theorem}
    Στο τέλος της ομιλίας της η ρεπόρτερ συνεχίζει να κοιτάει την κάμερα φορώντας ένα άβολο τρομακτικό χαμόγελο. Όταν επικρατήσει απόλυτη ησυχία στο δωμάτιο αρχίζουν να ηχούν σειρήνες. Ο ήχος τους αρχίζει να φαντάζει όλο ένα και πιο μακρινός. Επικεντρωνόμαστε στον Άρη. 


\newpage
\section{Sequence 3}
    
    \vspace{0.2cm}
    \begin{defn}[width=.95\textwidth]{Sequence Information}{}

        \vspace{0.1cm}
        \noindent $\blacktriangleright$ \textbf{ΘΕΜΑ Sequence:} Ending
        
        \vspace{0.1cm}
        \noindent $\blacktriangleright$ \textbf{Scene 1:} Απαιτείται δημιουργία του laptop hacking content πριν τα γυρίσματα

    \end{defn}

    
    \vspace{0.5cm}
    \subsection*{ \uline{\uline{\textbf{Scene 1} [Η Ανακάλυψη]}} } 
    \addcontentsline{toc}{section}{\protect\numberline{}3.1 \hspace{0.1cm} Scene 1 [Η Ανακάλυψη]}
    Βράδυ. Μεταβαίνουμε απότομα στο δωμάτιο του. Ο Άρης είναι ξαπλωμένος και μόλις ανοιξε τα μάτια του Σηκώνεται και πηγαίνει να κλείσει το παράθυρο επειδή κάνει κρύο. Έξω χιονίζει. Κάθεται στον υπολογιστή του να γράψει κώδικα κοιτώντας ένα χρονοδιάγραμμα που έχει χειρόγραμμένο στο γραφείο του. Μια σειρά από αριθμούς και δεδομένα εμφανίζονται ξαφνικά στην οθόνη.
    
   \begin{theorem}[width=1\textwidth]{Λόγια Άρη}{}
        
        \textbf{[A]} \hspace{0.2cm}  (ξαφνιασμένος): Τι στο…;
        
    \end{theorem}
    Το σύστημα γράφει και λέει:
    \begin{theorem}[width=1\textwidth]{AI  Announcement}{}
        
        \textbf{[A.I]} \hspace{0.2cm}  Πρόβλεψη ζωής: 98.7\%  ακρίβεια. Σπουδές → Μετριότητα. Καριέρα → Χωρίς εξέλιξη. Προσωπική ζωή → Μοναχική.
        
    \end{theorem}
    
    Ο Άρης κοιτάζει την οθόνη παγωμένος. Το μέλλον του είναι ήδη καθορισμένο.

    
    \vspace{0.5cm}
    \subsection*{ \uline{\uline{\textbf{Scene 2} [Η Απόφαση]}} } 
    \addcontentsline{toc}{section}{\protect\numberline{}3.2 \hspace{0.1cm} Scene 2 [Η Απόφαση]}
    
    Ο Άρης στο δωμάτιό του. Κοιτάζει την οθόνη, όπου το σύστημα συνεχίζει να αναλύει τη ζωή του.
    Με αποφασιστικότητα, σβήνει τα πάντα και αποσυνδέεται.
    Ένα τελευταίο μήνυμα εμφανίζεται: “Πρόβλεψη μη διαθέσιμη.”
    Ο Άρης χαμογελάει. Για πρώτη φορά, νιώθει πραγματικά ελεύθερος και έπειτα ανοίγει το βιβλίο 1984 του George Orwell και η κάμερα δείχνει με πρόσοψη το εξώφυλλο του βιβλίου και τον Άρη να το διαβάζει καθισμένος στην καρέκλα του γραφείου του.


\end{document}