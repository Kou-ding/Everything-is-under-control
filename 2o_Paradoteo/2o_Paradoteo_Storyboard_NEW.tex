\documentclass[hidelinks,titlepage,landscape, 12pt, a4paper]{article}


\usepackage{polyglossia}
\setdefaultlanguage{greek}
\setotherlanguage{english}

\usepackage{fontspec}
\setmainfont[
    Script=Greek,  % Ensures proper Greek rendering
    Ligatures=TeX, % Standard TeX ligatures (like `` and '')
]{Courier New}     % Main font for Greek and Latin text

% The "amsmath" package allows the inclusion of mathematica symbols in the LaTeX document
\usepackage{amsmath}
\usepackage{caption}
\usepackage{subcaption}
% The "array" package allows the center allignment of columns with wrapped content
\usepackage{array}
    \newcolumntype{P}[1]{>{\centering\arraybackslash}p{#1}}

%  The "ulem" package allows strikethrough effect in text
\usepackage{ulem}

% The "graphicx" package allows insertion of images
\usepackage{graphicx}

% The "subfig" packages allows the use of figures that consist of more than one images
\usepackage{subfig}

% The "wrapfig" package allows the insertion of images on the side of text and in other custom positions
\usepackage{wrapfig}

% The "ragged2e" package allows justified text (pliri stixoisi)
\usepackage[document]{ragged2e}

% The "xcolor" package allows colouring tables
\usepackage[table,xcdraw]{xcolor}

% The "hyperref" package allows adding links to LaTeX document
\usepackage{hyperref}

%  The "todonotes" package allows adding nice inline and side notes 
\usepackage[colorinlistoftodos]{todonotes}\setlength{\marginparwidth}{2.5cm}

% The "tcolorbox" package allows creating coloured boxes of text in the main body of text.
\usepackage[most]{tcolorbox}
        \newtcolorbox {theorem} [2] []{colback=green!5,colframe=green!35!black,colbacktitle=green!35!black,fonttitle=\bfseries,enhanced,attach boxed title to top left={yshift=-2mm,xshift=-7mm},width=.9\textwidth,arc=.7mm, title={#2},#1}%{th}
        
        \newtcolorbox {defn} [2] []{colback=blue!5,colframe=cyan!35!black,colbacktitle=blue!35!black,fonttitle=\bfseries,enhanced, attach boxed title to top left= {yshift=-2mm,xshift=-2mm}, title={#2},#1}%{def}

        \newtcolorbox {note} [2] []{colback=yellow!10!white,colframe=red!75!black,colbacktitle=red!75!black,fonttitle=\bfseries,enhanced, attach boxed title to top left= {yshift=-2mm,xshift=-2mm}, title={#2},#1}%{note}




%%%%%%%%%%%%%%%%%%%%%%%%%%%%%%%%%%%%%%%%%%
%%%%%%%%%% Page layout settings %%%%%%%%%%
% The "geometry" package allows the formating of the writable space in a page
\usepackage{geometry}
        \geometry{
            a4paper,
            total={150mm,247mm},
            left=30mm,
            top=30mm
        }

% The "parindent"  package allows setting the paragraph indent of the document (i.e. the empty space at the beggining of each paragraph)
\parindent=20pt
% By default, TeX indents the first line of each paragraphs by 20pt
\usepackage{parskip}
\parskip=3.5pt

% The "relsize"  package allows using font sizes larger than 'Huge'  and smaller than 'tiny'
\usepackage{relsize}
        %The "\renewcommand\RSlargest" command sets the maximum font size allowed by the 'relsize '  package
        \renewcommand\RSlargest{75pt}

%%%%%%%%%%  End - Page layout settings %%%%%%%%%% 
%%%%%%%%%%%%%%%%%%%%%%%%%%%%%%%%%%%%%%%%%%%%%%%%%

% The "float" package allows the placement of figures, images etc. in specific places in between text
\usepackage{float}

% The "afterpage" package allows the insertion of blank pages (eg. a blank page after the 'titlepage')
\usepackage{afterpage}
        \newcommand\blankpage{%
            \null
            \thispagestyle{empty}%
            %\addtocounter{page}{-1}%
            \newpage}


% The "mathtools"  package allows  the insertion of text, formulas etc. above arrows (=> etc.) in mathematical expressions
\usepackage{mathtools}

% The "mathrsfs" package allows the use of curly math letters, like the Lagrange 'L'
\usepackage{ mathrsfs }

% The "cancel" package allows diagonal arrow-kind-strikethrough in math expressions
        %\usepackage{cancel}
\usepackage[makeroom]{cancel}
    \makeatletter
    % #1, #2 offset of label   #6 extra width to clear arrowhead
    % #3, #4 vector direction  #7 superscript label style
    % #5 vector width          #8 superscript label
    \def\cantox@vector#1#2#3#4#5#6#7#8{%
        \dimen@.5\p@
        \setbox\z@\vbox{\boxmaxdepth.5\p@
        \hbox{\kern-1.2\p@\kern#1\dimen@$#7{#8}\m@th$}}%
    \ifx\canto@fil\hidewidth  \wd\z@\z@ \else \kern-#6\unitlength \fi
        \ooalign{%
        \canto@fil$\m@th \CancelColor
        \vcenter{\hbox{\dimen@#6\unitlength \kern\dimen@
            \multiply\dimen@#4\divide\dimen@#3 \vrule\@depth\dimen@\@width\z@
            \vector(#3,-#4){#5}%
    }}_{\raise-#2\dimen@\copy\z@\kern-\scriptspace}$%
        \canto@fil \cr
        \hfil \box\@tempboxa \kern\wd\z@ \hfil \cr}}
    \def\bcancelto#1#2{\let\canto@vector\cantox@vector\cancelto{#1}{#2}}
    \makeatother

% The "tabto" package allows the use of the regular 'Tab' funcionality of inserting empty space
\include{tabto}


% The "amssymb" package allows the use of yet more mathematical symbols
\usepackage{amssymb}

%The "enumitem" package allows the use of lists numbered alphabetically etc.
\usepackage{enumitem}

%The "diagbox", "color", "vcell", "adjustbox" packages allow custom adjustments to tables
\usepackage{diagbox}
\usepackage{color}
\usepackage{vcell}
\usepackage{adjustbox}

%The "changepage" package allows the ????????
\usepackage{changepage} 

%The "unicode-math" package allows the use of more Greek symbols in equations
\usepackage{unicode-math}

%tables
\usepackage{multirow}
\usepackage[table,xcdraw]{xcolor}
\usepackage{colortbl}

%The following two commands allow the numbering of \paragraphs and their inclusion in the Table of Contents.
% further explanation: https://tex.stackexchange.com/questions/186981/is-there-a-subsubsubsection-command
\setcounter{tocdepth}{4}
\setcounter{secnumdepth}{4}

%The "gensymb" package allows the insertion of the 'degree' symbol
\usepackage{gensymb}

%The "xfrac" package allows the insertion of 'diagonal' fraction 
\usepackage{xfrac}

%The "multicol" package allows the insertion of lists with more than one columns
\usepackage{multicol}


\usepackage{float}  % Adds the [H] positioning specifier

\begin{document}

\begin{titlepage}
\centering  
\begin{center}
    
    \begin{figure}[H]
        \vspace{-1cm}
        \centering
        \includegraphics[width=8cm]{Images/AUTH_Banner.png}
    \end{figure}

    \vspace{1cm}               
    \Huge
    
    \hspace{-1cm}\textlarger[4]{ Τ}\textlarger[1.5]{ΗΛΕΟΠΤΙΚΑ }\textlarger[4]{Σ}\textlarger[1.5]{ΥΣΤΗΜΑΤΑ}  
    
            
    \vspace{1cm}
    \Huge
    \textbf{"ΟΛΑ ΕΙΝΑΙ ΥΠΟ ΕΛΕΓΧΟ"}
    
    \LARGE
    \vspace{1cm}
    \textbf{2ο Παραδοτέο}\\
    - Storyboard -
            
    \vspace{1cm}
    \large
    \vspace{0.2cm}
    Καρατζάς Αθανάσιος, ΑΕΜ 10514\\
    \vspace{0.2cm}
    Παναπακίδης Δημήτριος, ΑΕΜ 9298\\
    \vspace{0.2cm}
    Παπαδάκης Κωνσταντίνος Φώτιος, ΑΕΜ 10371\\
    \vspace{0.2cm}
    Παρασκευοπούλου Άννα, ΑΕΜ 10238\\
    \vspace{0.2cm}
    Χατζηγεωργίου Αδάμ, ΑΕΜ 9844 
            
    \large
    Τμήμα Ηλεκτρολόγων Μηχανικών και Μηχανικών Υπολογιστών ΑΠΘ\\
    Δευτέρα 31 Μαρτίου 2025
            
\end{center}
\date{Εαρινό Εξάμηνο 2024-2025}

%\afterpage{\blankpage} % Inserting a blank page after titlepage

\end{titlepage}

\justifying
\setcounter{page}{2} % Because this is actually page 3
%\afterpage{\blankpage} % Inserting a blank page

\newpage

\begin{figure}[H]

    \includegraphics[width=30.5cm]{Images/Storyboard_1.jpg}
    \label{fig:Storyboard_1}
\end{figure}



\newpage

\begin{figure}[H]
    \vspace{-2.5cm}
    \hspace{-3.5cm}
    \includegraphics[width=30.5cm]{Images/Storyboard_2.jpg}
    \label{fig:Storyboard_2}
\end{figure}



\newpage

\begin{figure}[H]
    \vspace{-2.5cm}
    \hspace{-3.5cm}
    \includegraphics[width=30.5cm]{Images/Storyboard_3.jpg}
    \label{fig:Storyboard_3}
\end{figure}

\newpage

\begin{figure}[H]
        \vspace{-2.5cm}
        \hspace{-3.5cm}
        \includegraphics[angle= -90, width=20cm]{Images/Storyboard_4.jpg}
        \label{fig:Storyboard_4}
    \end{figure}



    \newpage

\begin{figure}[H]
        \vspace{-2.5cm}
        \hspace{-3.5cm}
        \includegraphics[angle= -90, width=20cm]{Images/Storyboard_5.jpg}
        \label{fig:Storyboard_5}
    \end{figure}




    \newpage

\begin{figure}[H]
        \vspace{-2.5cm}
        \hspace{-3.5cm}
        \includegraphics[angle= -90, width=20cm]{Images/Storyboard_6.jpg}
        \label{fig:Storyboard_6}
    \end{figure}



    \newpage

\begin{figure}[H]
        \vspace{-2.5cm}
        \hspace{-3.5cm}
        \includegraphics[angle= -90, width=20cm]{Images/Storyboard_7.jpg}
        \label{fig:Storyboard_7}
    \end{figure}



    \newpage

\begin{figure}[H]
        \vspace{-2.5cm}
        \hspace{-3.5cm}
        \includegraphics[angle= -90, width=20cm]{Images/Storyboard_8.jpg}
        \label{fig:Storyboard_8}
    \end{figure}




    \newpage

\begin{figure}[H]
        \vspace{-2.5cm}
        \hspace{-3.5cm}
        \includegraphics[angle= -90, width=20cm]{Images/Storyboard_9.jpg}
        \label{fig:Storyboard_9}
    \end{figure}



    \newpage

\begin{figure}[H]
        \vspace{-2.5cm}
        \hspace{-3.5cm}
        \includegraphics[angle= -90, width=20cm]{Images/Storyboard_10.jpg}
        \label{fig:Storyboard_10}
    \end{figure}



    \newpage

\begin{figure}[H]
        \vspace{-1cm}
        \hspace{-1.5cm}
        \includegraphics[angle= -90, width=18cm]{Images/Storyboard_11.jpg}
        \label{fig:Storyboard_11}
    \end{figure}



    \newpage

\begin{figure}[H]
        \vspace{-2.5cm}
        \hspace{-3.5cm}
        \includegraphics[angle= -90, width=20cm]{Images/Storyboard_12.jpg}
        \label{fig:Storyboard_12}
    \end{figure}



    \newpage

\begin{figure}[H]
        \vspace{-2.5cm}
        \hspace{-3.5cm}
        \includegraphics[angle= -90, width=20cm]{Images/Storyboard_13.jpg}
        \label{fig:Storyboard_13}
    \end{figure}



    \newpage

\begin{figure}[H]
        \vspace{-2.5cm}
        \hspace{-3.5cm}
        \includegraphics[angle= -90, width=20cm]{Images/Storyboard_14.jpg}
        \label{fig:Storyboard_14}
    \end{figure}



    \newpage

\begin{figure}[H]
        \vspace{-2.5cm}
        \hspace{-3.5cm}
        \includegraphics[angle= -90, width=20cm]{Images/Storyboard_15.jpg}
        \label{fig:Storyboard_15}
    \end{figure}







\end{document}