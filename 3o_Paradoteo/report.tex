\documentclass[hidelinks, titlepage, 12pt, a4paper]{article}

%%%%%%%%%%%%%%%% Text structure %%%%%%%%%%%%%%%%
\usepackage{polyglossia}
\setdefaultlanguage{greek}
\setotherlanguage{english}
\usepackage{fontspec}
\setmainfont{FreeMono}
% The "parindent"  package allows setting the paragraph indent of the document (i.e. the empty space at the beggining of each paragraph)
\parindent=20pt
% By default, TeX indents the first line of each paragraphs by 20pt
\usepackage{parskip}
\parskip=3.5pt
% The "relsize"  package allows using font sizes larger than 'Huge'  and smaller than 'tiny'
\usepackage{relsize}
    %The "\renewcommand\RSlargest" command sets the maximum font size allowed by the 'relsize '  package
    \renewcommand\RSlargest{75pt}
% The "graphicx" package allows insertion of images
\usepackage{graphicx}
% The "xcolor" package allows colouring tables
\usepackage[table,xcdraw]{xcolor}
% The "hyperref" package allows adding links to LaTeX document
\usepackage{hyperref}
% The "ragged2e" package allows justified text (pliri stixoisi)
\usepackage[document]{ragged2e}
% Allows for specific itemize bullet shapes
\usepackage{enumitem}
% The "geometry" package allows the formating of the writable space in a page
\usepackage{geometry}
\geometry{
    a4paper,
    total={150mm,247mm},
    left=20mm,
    top=20mm,
    right=20mm,
    bottom=20mm
}
\usepackage{titlesec} % Required for customizing section titles
\titleformat{\section}{\normalfont\bfseries\large}{}{0em}{}
\titleformat{\subsection}{\normalfont\bfseries\large}{}{0em}{}
\titleformat{\subsubsection}{\normalfont\bfseries\normalsize}{}{0em}{}

%%%%%%%%%%%%%%%% Math %%%%%%%%%%%%%%%%
% The "float" package allows the placement of figures, images etc. in specific places in between text
\usepackage{float}
% The "mathtools" package allows the insertion of text, formulas etc. above arrows (=> etc.) in mathematical expressions
\usepackage{mathtools}
% The "amsmath" package allows the inclusion of mathematica symbols in the LaTeX document
\usepackage{amsmath}
% The "mathrsfs" package allows the use of curly math letters, like the Lagrange 'L'
\usepackage{mathrsfs}
% The "amssymb" package allows the use of yet more mathematical symbols
\usepackage{amssymb}

\begin{document}

\begin{titlepage}
\begin{center}
    \vspace{1cm}

    \begin{figure}[h]
        \centering
        \includegraphics[width=8cm]{Images/AUTH_Banner.png}
    \end{figure}

    \vspace{2.5cm}               
    \Huge
    
    \hspace{-1cm}\textlarger[4]{ Τ}\textlarger[1.5]{ΗΛΕΟΠΤΙΚΑ }\textlarger[4]{Σ}\textlarger[1.5]{ΥΣΤΗΜΑΤΑ}  
    
            
    \vspace{2.5cm}
    \Huge
    \textbf{"ΟΛΑ ΕΙΝΑΙ ΥΠΟ ΕΛΕΓΧΟ"}
    
    \LARGE
    \vspace{2cm}
    \textbf{3ο Παραδοτέο}\\
    - Movie Report -
            
    \vspace{2cm}
    \Large
    \vspace{0.2cm}
    Καρατζάς Αθανάσιος, ΑΕΜ 10514\\
    \vspace{0.2cm}
    Παναπακίδης Δημήτριος, ΑΕΜ 9298\\
    \vspace{0.2cm}
    Παπαδάκης Κωνσταντίνος Φώτιος, ΑΕΜ 10371\\
    \vspace{0.2cm}
    Παρασκευοπούλου Άννα, ΑΕΜ 10238\\
    \vspace{0.2cm}
    Χατζηγεωργίου Αδάμ, ΑΕΜ 9844
            
    \vfill
    \Large       
            
    \large
    Τμήμα Ηλεκτρολόγων Μηχανικών και Μηχανικών Υπολογιστών ΑΠΘ\\
    Δευτέρα 31 Μαρτίου 2025
            
\end{center}
\date{\today}

\end{titlepage}

\tableofcontents

\newpage
    
\section{Σενάριο}
\subsection{Sequence 1}

    \subsubsection{Scene 1: Δρόμος της επιστροφής} 
    Η φοιτήτρια βρίσκεται στον δρόμο της επιστροφής από το φοιτητικό πάρτι. ( Το σπίτι βρίσκεται πολύ κοντά στα πανεπιστήμια) Έχει πιει και για αυτό η κάμερα είναι out of focus καθ' όλη τη διάρκεια της λήψης έξω από το σπίτι της στο δρόμο της επιστροφής.
 
    \vspace{0.5cm}
    
    \subsubsection{Scene 2: Σπίτι} 
    Η κάμερα δυσκολεύεται να εστιάσει (ακόμα out of focus γιατί η φοιτήτρια είναι πιωμένη) αλλά προλαβαίνει να κάνει μια γρήγορη εστίαση έτσι ώστε η κοπέλα να μπορέσει να βάλει το κλειδί στην κλειδαριά και να μπει μέσα στο σπίτι. Όταν μπαίνει μέσα ανοίγει την τηλεόραση όπως κάνει κάθε βράδυ για να την πάρει ο ύπνος αλλά είναι τόσο κουρασμένη που δεν προλαβαίνει να ακούσει, ούτε καν να νανουριστεί από την τηλεόραση. Καθώς μπαίνει όλο και περισσότερο σε μια ενδιάμεση κατάσταση μεταξύ ύπνου και ξύπνιου την παίρνει τελικά ο ύπνος και μαζί χάνεται και κάθε ήχος εντός του δωματίου. Αφού κλείσει τα μάτια της προλαβαίνουμε να δούμε στην τηλεόραση να παίζουν Breaking News. Ο ήχος δεν είναι καθαρός και το μόνο που μισο-φαίνεται είναι ο τίτλος των νέων αυτών.

\subsection{Sequence 2}

    \vspace{0.5cm}
    \subsubsection{Scene 1: Πρωινό ξύπνημα} 
    Το ξυπνητήρι φέρνει το επόμενο πρωί όπου η τηλεόραση έχει κλείσει λόγω του χρονοδιακόπτη και η πρωταγωνίστρια ξεκινά μισοκοιμισμένη να ετοιμάζεται για να πάει στη σχολή. Σηκώνεται, βάζει τις παντόφλες της, πλένει τα δόντια της, μεταβαίνει ντυμένη μέχρι την κουζίνα, παίρνει ένα σνακ και βγαίνει από την εξώπορτα.

    \vspace{0.5cm}
    
    \subsubsection{Scene 2: Στο δρόμο για  τη σχολή} 
    Έπονται πλάνα από τον δρόμο προς το πανεπιστήμιο όπου, φορώντας ακουστικά στα αυτιά, παρατηρεί κόσμο να είναι πολύ βιαστικός και ανήσυχος. Το περπάτημα της όμως είναι γρήγορο και η ίδια δεν δίνει πολύ σημασία σε κανέναν. Φτάνει στο μετρό το οποίο παίρνει μέχρι τον σταθμό "Πανεπιστήμιο". Η μουσική που σταδιακά κλιμακώνει το μυστήριο.

    \vspace{0.5cm}
    
    \subsubsection{Scene 3: Άφιξη στη σχολή}
    Όταν φτάνει στο πανεπιστήμιο βλέπει πολύ λιγότερα άτομα από όσα είχαν έρθει την πρώτη μέρα στη σχολή. Τα πρόσωπα που την περιτριγυρίζουν είναι στο κινητό τους και αρκετά ανήσυχα.

    \vspace{0.5cm}
    \subsubsection{Scene 4: Πρώτη αλληλεπίδραση} 
    Ένας τύπος την πλησιάζει και της λέει:
    \\
    Διάλογος 
    \begin{itemize}[label=-]
        \item \textbf{[A]} \hspace{0.2cm} Hey! Μήπως θα μπορούσα να χρησιμοποιήσω το κινητό σου...
        \item \textbf{[Γ]} \hspace{0.2cm} Ναι, κανένα θέμα
        \item \textbf{[Α]} \hspace{0.2cm} Ευχαριστώ πολύ
    \end{itemize}

    Ο τύπος πληκτρολογεί έναν αριθμό και βάζει το κινητό στο αυτί του
    \\
    Διάλογος
    \begin{itemize}[label=-]
        \item \textbf{[Α]} \hspace{0.2cm}  (στο τηλέφωνο) Έλα σε παίρνω από μια συμφοιτήτρια, πως είναι ο μπαμπάς? 
        \item \textbf{[Μ]} \hspace{0.2cm}  (Μητέρα αχνά) Δεν δουλεύει η πιστωτική κάρτα... Οι γιατροί αρνούνται να τον χειρουργήσουν χωρίς να πληρώσουμε όλο το ποσό. Προσπαθώ όλο το πρωί σε όλα τα ATM και κανένα δεν λειτουργεί, δεν ξέρω τι να κάνω... Έλα σε παρακαλώ να με βοηθήσεις. Ίσως κάνω κάτι λάθος?... (συναισθηματικό ύφος, αμήχανο γέλιο)
        \item \textbf{[Α]} \hspace{0.2cm}  (μετά από 3 δευτερόλεπτα σιγή) Που είσαι για να σε βρω?  
        \item \textbf{[Μ]} \hspace{0.2cm}  Στο νοσοκομείο Παναγία.     
        \item \textbf{[Α]} \hspace{0.2cm}  Εντάξει παίρνω ταξί και έρχομαι.    
    \end{itemize}   
        
    \vspace{0.2cm}
    Έπειτα ο τύπος έχει ένα κενό βλέμμα για 5 δευτερόλεπτα και γυρνάει προς την κοπέλα με αργό βάδην λέγοντας:
    \vspace{0.2cm}


    Διάλογος
    \begin{itemize}[label=-]
        \item \textbf{[Α]} \hspace{0.2cm}  Ευχαριστώ πολύ (με ένα ψεύτικο χαμόγελο)
    \end{itemize}
    
    \vspace{0.5cm}
    
    \subsubsection{Scene 5: Δεύτερη αλληλεπίδραση} 
    Η προσοχή της κοπέλας στρέφεται για λίγα δευτερόλεπτα μερικά θρανία πιο μπροστά σε μερικούς συμφοιτητές. Η εναλλαγή της κάμερας είναι γρήγορη, ελαφρώς τρεμάμενη και θολή στις άκρες σαν να προσομοιάζει τις γρήγορες ματιές που ρίχνει και η πρωταγωνίστρια.
    
    Διάλογος
    \begin{itemize}[label=-]
         \item \textbf{[Τ1]} \hspace{0.2cm}  Προσπαθώ να μπω στο sis να δω αν βγήκε βαθμολογία αλλά όταν κάνω login η σελίδα είναι όλη λευκή. Ξέρεις τι παίζει?
         \item \textbf{[Τ2]} \hspace{0.2cm}  Δεν έχω ιδέα κάτι θα κάνουμε πάλι στο ΚΗΔ. Τους ξέρεις πως είναι χαχα, είμαστε οι beta testers τους.
         \item \textbf{[Τ1]} \hspace{0.2cm}  Χάρη μου κάνουν. Παρατείνουν την ευδαιμονία της άγνοιας.
    \end{itemize}
    
    \vspace{0.5cm}
    
    \subsubsection{Scene 6: Τρίτη αλληλεπίδραση} 
    Εναλλαγή στο επόμενο άτομο. Πάει να συνδεθεί σε ένα παιχνίδι στο κινητό του και οι κωδικοί του δεν είναι σωστοί.

    \vspace{0.5cm}
    
    \subsubsection{Scene 7: Τέταρτη αλληλεπίδραση} 
    Εναλλαγή στο επόμενο άτομο (Άρης). Κάθεται στην άκρη της αίθουσας και διαβάζει ένα βιβλίο.
    
    \subsubsection{Scene 8: Ειδοποίηση} 
    Εναλλαγή στο επόμενο άτομο. Το κινητό της χτυπάει σαν τα μηνύματα της πολιτικής προστασίας με έναν δυνατό θόρυβο όμοιο με αυτό της πολιτικής προστασίας. Το ίδιο γίνεται και σε όλα τα υπόλοιπα άτομα της αίθουσας. Όλοι οι ήχοι από τα κινητά των συμμαθητών μειώνονται από την δική της αντίληψη και γίνονται θολοί. Η κοπέλα εστιάζει στην οθόνη του κινητού.

    Mobile AI announcement
    \begin{itemize}[label=-] 
        \item \textbf{[Κ]} \hspace{0.2cm}  (Κινητό AI voice) Από τις πρόσφατες συνομιλίες σου αποδεικνύεται η αντιγραφή σου στο μάθημα Πεδίο 2. Αυτό επηρεάζει τον κοινωνικό σου σκορ. (παύση ενός δευτερολέπτου)
    \end{itemize}
    
    Το σκορ πέφτει από 3.2 σε 2.7. Και γίνεται ένα γρήγορο flashback στο όταν αντέγραψε. Γίνεται χρήση ενός faded σχεδον ασπρομαυρου φιλτρου για να υποδηλώσουμε ότι έγινε στο παρελθον.

    Mobile AI announcement
    \begin{itemize}[label=-] 
        \item \textbf{[Κ]} \hspace{0.2cm}  Προσοχή! Αν το σκορ σου πέσει κάτω από 2.5 θα υπάρξουν αυστηρές κοινωνικοοικονομικές κυρώσεις! Ελπίζουμε να γίνεις και εσύ κομμάτι ενός δίκαιου και ασφαλούς μέλλοντος! :)
    \end{itemize}
    Όλοι οι ήχοι πλέον από θολοί γίνονται ένα βουητό καθώς το βλέμμα της γίνεται όλο και πιο έντονο. 

    \vspace{0.5cm}
     
    \subsubsection{Scene 9: Breaking News στο λάπτοπ ενός φοιτητή}
    Η πρωταγωνίστρια σαστισμένη αναζητά απαντήσεις στο περιβάλλον της και παρατηρεί πως όσο ήταν απορροφημένη στο κινητό της οι υπόλοιποι φοιτητές είχαν ήδη μαζευτεί μπροστά από μια οθόνη λάπτοπ. Καθώς πλησιάζει με διστακτικό βήμα, σιγά-σιγά, ακούγεται μια reporter να ανακοινώνει τα νέα.

    Λόγια παρουσιάστριας
    \begin{itemize}[label=-]
        \item \textbf{[Ν]} \hspace{0.2cm}  (news reporter) Η Βαϊρέλια (μία πλασματική χώρα) έχει εξαπολύσει μια παγκόσμιας κλίμακας κυβερνοεπίθεση χάρη στη πρόσφατη ανάπτυξη ενός κβαντικού υπολογιστή ικανού να σπάσει την κρυπτογράφηση δημοσίου κλειδιού RSA. Ακόμη δεν ξέρουμε το πλήρες εύρος της επίθεσης και μένουμε διαρκώς σε επαφή με ειδικούς ενημερώνοντας σας συνεχώς για την κατάσταση των  πληροφοριακών συστημάτων εντός της χώρας καθώς και σε παγκόσμιο επίπεδο. Προς το παρόν έχουν επιβεβαιωθεί μαζικές αποκρυπτογραφήσεις δεδομένων, hacking υποδομών συνδεδεμένες με τον διαδίκτυο, μαζικές κρυπτογραφήσεις και διαγραφές δεδομένων. Ο στόχος αυτών δεν είναι πλήρως γνωστός μέχρι στιγμής ενώ δεν έχει διερευνηθεί ακόμα ποια συστήματα έχουν επηρεαστεί με τι τρόπο. Το ίντερνετ δεν είναι ασφαλές Βεβαιωθείτε ότι ενημερώνεστε από αναλογικές μορφές ενημέρ-
    \end{itemize}
    Η εικόνα παγώνει για λίγο εμφανίζοντας στην οθόνη glitches, RGB noise και η χροιά της φωνής αλλάζει ανεπαίσθητα. Η ρεπόρτερ πλέον χαμογελάει.
    
    Λόγια παρουσιάστριας
    \begin{itemize}[label=-]
        \item \textbf{[Ν]} \hspace{0.2cm}  (news reporter) Όλα είναι υπό έλεγχο. Δεν υπάρχει κανένας λόγος ανησυχίας.   
    \end{itemize}
    Στο τέλος της ομιλίας της η ρεπόρτερ συνεχίζει να κοιτάει την κάμερα φορώντας ένα άβολο τρομακτικό χαμόγελο. Όταν επικρατήσει απόλυτη ησυχία στο δωμάτιο αρχίζουν να ηχούν σειρήνες. Ο ήχος τους αρχίζει να φαντάζει όλο ένα και πιο μακρινός. Επικεντρωνόμαστε στον Άρη. 
    
\subsection{Sequence 3}
    
    \subsubsection{Scene 1: Η Ανακάλυψη} 
    Βράδυ. Μεταβαίνουμε απότομα στο δωμάτιο του. Ο Άρης είναι ξαπλωμένος και μόλις ανοιξε τα μάτια του Σηκώνεται και πηγαίνει να κλείσει το παράθυρο επειδή κάνει κρύο. Έξω χιονίζει. Κάθεται στον υπολογιστή του να γράψει κώδικα κοιτώντας ένα χρονοδιάγραμμα που έχει χειρόγραμμένο στο γραφείο του. Μια σειρά από αριθμούς και δεδομένα εμφανίζονται ξαφνικά στην οθόνη.

    Λόγια Άρη
    \begin{itemize}[label=-]    
        \item \textbf{[A]} \hspace{0.2cm}  (ξαφνιασμένος): Τι στο…;        
    \end{itemize}
    Το σύστημα γράφει και λέει:

    AI  Announcement
    \begin{itemize}[label=-]
        \item \textbf{[A.I]} \hspace{0.2cm}  Πρόβλεψη ζωής: 98.7\%  ακρίβεια. Σπουδές → Μετριότητα. Καριέρα → Χωρίς εξέλιξη. Προσωπική ζωή → Μοναχική.    
    \end{itemize}
    Ο Άρης κοιτάζει την οθόνη παγωμένος. Το μέλλον του είναι ήδη καθορισμένο.

    \vspace{0.5cm}
    
    \subsubsection{Scene 2: Η Απόφαση}     
    Ο Άρης στο δωμάτιό του. Κοιτάζει την οθόνη, όπου το σύστημα συνεχίζει να αναλύει τη ζωή του.
    Με αποφασιστικότητα, σβήνει τα πάντα και αποσυνδέεται.
    Ένα τελευταίο μήνυμα εμφανίζεται: “Πρόβλεψη μη διαθέσιμη.”
    Ο Άρης χαμογελάει. Για πρώτη φορά, νιώθει πραγματικά ελεύθερος και έπειτα ανοίγει το βιβλίο 1984 του George Orwell και η κάμερα δείχνει με πρόσοψη το εξώφυλλο του βιβλίου και τον Άρη να το διαβάζει καθισμένος στην καρέκλα του γραφείου του.
    
\newpage

\subsection{Σχολιασμός Συλλογιστικής Διαδικασίας}
    Το σενάριο μας βασίζεται σε δύο παράλληλους άξονες. Τον πραγματικό κόσμο και τον κόσμο των ιδεών. Στον πραγματικό κόσμο παρατηρούμε σε υπερβολικό βαθμό, για λόγους έμφασης, τις συνοπτικές αλληλεπιδράσεις των χαρακτήρων μας με μια ανερχόμενη πολιτική δύναμη η οποία κατέχει μέσα παγκόσμιας παρακολούθησης. Μιας δύναμης που έχει επιτύχει στην δημιουργία ενός κβαντικού υπολογιστή με αρκετά κβαντικά bits και αρκετή ανθεκτικότητα στον εγγενή θόρυβο των κβαντικών υπολογιστών. Κατ' επέκταση οι βασικές μορφές κρυπτογράφησης και διασφάλισης ασφαλούς επικοινωνίας μεταξύ κόμβων του διαδικτύου έχουν λυθεί στο φως ενός καινούριου τρόπου επιτέλεσης υπολογισμών. Ενός νέου είδους κβαντικών κυκλωμάτων, και των συνεπαγόμενων κβαντικών τους ιδιοτήτων, που δίνουν ζωή σε καινούριους αλγορίθμους οι οποίοι καθιστούν προβλήματα, παλαιότερα άλυτα σε λογικά χρονικά πλαίσια, πλέον τετριμμένα. Σε έναν κόσμο όπου η ζωή μας είναι τόσο ένσαρκη όσο και ψηφιακή η δύναμη που αποκτά αυτό το φανταστικό κράτος είναι απαράμιλλη και ανήκουστη καθώς η πλήρης διασυνδεσιμότητα του διαδικτύου συνεπάγεται μέγιστη επιρροή στις ζωές όλων.
    
    \vspace{0.5cm}
    
    Στον κόσμο των ιδεών, από την άλλη πλευρά, προβληματιζόμαστε μαζί με τον παράλληλο αφηγητή της ταινίας και πρωταγωνιστή Άρη για θέματα σχετικά με το ισοζύγιο ασφάλειας και ελευθερίας στα πλαίσια του διαδικτύου. Παρατηρούμε στενά την συλλογική διαδικασία του και τα συμπεράσματα στα οποία οδηγείται επηρεάζοντας την στάση του στην ταινία.
    
    \vspace{0.5cm}
    
    Η ταινία ξεκινάει με την πρωταγωνίστρια μας "Άννα" να επιστρέφει σπίτι, νύχτα, έπειτα από βραδινή της έξοδο με δύο φίλες της. Η κατάσταση μέθης δικαιολογεί την απρόσεκτη συμπεριφορά της η οποία οδηγεί εν τέλει, το κινητό της και το τηλεκοντρόλ της τηλεόρασης στο πάτωμα. Η ιδέα ότι το δελτίο ειδήσεων έπιασε τον κόσμο στον ύπνο του μας δίνει τη δυανατότητα να αποτυπώσουμε έπειτα στον φακό τις πρωτόγνωρες αντιδράσεις των φοιτητών που δεν αντιλήφθηκαν το φαινόμενο και πήγαν την επόμενη μέρα στη σχολή.
    
    \vspace{0.5cm}
    
    Έπειτα σιγά σιγά αρχίζουμε να εστιάζουμε στις σφοδρές συνέπειες του φαινομένου μέσω των ματιών της πρωταγωνίστριας. Η κατάρρευση του ψηφιακού χρηματοπιστωτικού συστήματος συναλλαγών, οι μαζικές τροποποιήσεις, κλοπές και διαγραφές δεδομένων καθώς και η μεταβλητή επίδραση στη κανονική λειτουργία των server που όλα τα παραπάνω συνεπάγονται, θέτουν το κλίμα ορίζοντας κάποιες επιφανειακές, καθαρά λειτουργικές, επιδράσεις που είχε η νέα τεχνολογία στον κόσμο. Καθώς περνούν τα λεπτά ο θεατής όμως καταλαβαίνει ότι η επιρροή αυτή είναι πολύ πιο βαθειά, επηρεάζοντας τα μέσα μαζικής ενημέρωσης, όπως βλέπουμε στην απότομη αλλαγή του ύφους της δημοσιογράφου, καθώς και το ίδιο το παγκόσμιο νομοθετικό και δικαστικό σύστημα μιας πλήρους ψηφιακά διασυνδεδεμένης κοινωνίας όπου δεν είναι προϋπόθεση η γεωγραφική εγγύτητα πλέον για την εφαρμογή νόμων.
    
    \vspace{0.5cm}

    %%%%% Θαναση βαλε το δικο σου κομματι εδώ %%%%%%
    ερμηνεια εδ
    %%%%%%%%%%%%%%%%%%%%%%%%%%%%%%%%%%%%%%%%%%%%%%%%%
    
    \vspace{0.5cm}
    Καθ'όλη τη ταινία η "Άννα" αποτελεί τα μάτια της ιστορίας μας καθώς από τα δικά της μάτια εκτυλίσσεται η ιστορία ενώ ο "Άρης" αποτελεί το μυαλό-φωνή της ιστορίας μας. Και οι δύο μαζί αποτελούν τους φορείς της ιστορίας μας. Ο θεατής βλέπει τον κόσμο με τα μάτια της "Άννας" ενώ σκέφτεται τα ίδια πράγματα που προβληματίζουν τον "Άρη".

\subsection{Προσαρμοστικές τροποποιήσεις}

\subsubsection{Κορεσμός σε βουβές σκηνές}
    Η αρχική μας προσέγγιση περιείχε μόνο την παράλληλη της πραγματικότητας, όμως οι πολλές βουβές σκηνές σε μία ταινία μικρού μήκους σε συνδυασμό με το σχόλιο του κ.Σεβαστιάδη: "Η ταινία πρέπει να έχει κάτι να πει στον θεατή", μας ώθησε στο να δημιουργήσουμε μία παράλληλη φωνή η οποία θα σχολιάζει θέματα κυβερνοασφάλειας από μια τόσο φιλοσοφική όσο πολιτική σκοπιά. Αποδώσαμε τον ρόλο του αφηγητή στον υφιστάμενο χαρακτήρα της ταινίας μας "Άρη" ο οποίος δείχνει να έχει ήδη μια κριτική στάση απέναντι στα δρώμενα. Έτσι ενσαρκώνουμε τις ιδεολογικές του απόψεις υπό την μορφή μονολόγου ο οποίος θέτει τον θεατή σε σκέψεις ενώ έπειτα τοποθετείται υπέρ της ανωνυμίας στο διαδίκτυο.

\subsubsection{Βασική ρύθμιση Τηλεόρασης}
    Κατά τη διάρκεια των σκηνών συνειδητοποιήσαμε ότι δεν βγάζει νόημα η πρωταγωνίστρια να μπορεί να ανοίξει την τηλεόραση συνειδητά στην κατάσταση που βρίσκεται. Οπότε τροποποιήσαμε το σενάριο να ρίχνει κατά λάθος το τηλεκοντρόλ στο πάτωμα όπου πατιέται και προβάλλεται το δελτίο ειδήσεων.

\section{Σκηνοθεσία}
\subsection{Storyboard}
    
    \begin{figure}[H]
        \vspace{-0.63cm}
        \hspace{-1.8cm}
        \includegraphics[angle= -90, width=1.05\textwidth]{Images/Storyboard_1.jpg}
        \label{fig:Storyboard_1}
    \end{figure}

    \newpage

    \begin{figure}[H]
        \vspace{-0.63cm}
        \hspace{-1.8cm}
        \includegraphics[angle= -90, width=1.05\textwidth]{Images/Storyboard_2.jpg}
        \label{fig:Storyboard_2}
    \end{figure}

    \newpage

    \begin{figure}[H]
        \vspace{-0.63cm}
        \hspace{-1.8cm}
        \includegraphics[angle= -90, width=1.05\textwidth]{Images/Storyboard_3.jpg}
        \label{fig:Storyboard_3}
    \end{figure}

    \newpage

    \begin{figure}[H]
        \vspace{-0.63cm}
        \hspace{-1.8cm}
        \includegraphics[angle= -90, width=1.05\textwidth]{Images/Storyboard_4.jpg}
        \label{fig:Storyboard_4}
    \end{figure}

    \newpage

    \begin{figure}[H]
        \vspace{-0.63cm}
        \hspace{-1.8cm}
        \includegraphics[angle= -90, width=1.05\textwidth]{Images/Storyboard_5.jpg}
        \label{fig:Storyboard_5}
    \end{figure}

    \newpage

    \begin{figure}[H]
        \vspace{-0.63cm}
        \hspace{-1.8cm}
        \includegraphics[angle= -90, width=1.05\textwidth]{Images/Storyboard_6.jpg}
        \label{fig:Storyboard_6}
    \end{figure}

    \newpage

    \begin{figure}[H]
        \vspace{-0.63cm}
        \hspace{-1.8cm}
        \includegraphics[angle= -90, width=1.05\textwidth]{Images/Storyboard_7.jpg}
        \label{fig:Storyboard_7}
    \end{figure}

    \newpage

    \begin{figure}[H]
        \vspace{-0.63cm}
        \hspace{-1.8cm}
        \includegraphics[angle= -90, width=1.05\textwidth]{Images/Storyboard_8.jpg}
        \label{fig:Storyboard_8}
    \end{figure}

    \newpage
    
    \begin{figure}[H]
        \vspace{-0.63cm}
        \hspace{-1.8cm}
        \includegraphics[angle= -90, width=1.05\textwidth]{Images/Storyboard_9.jpg}
        \label{fig:Storyboard_9}
    \end{figure}
    
    \newpage

    \begin{figure}[H]
        \vspace{-0.63cm}
        \hspace{-1.8cm}
        \includegraphics[angle= -90, width=1.05\textwidth]{Images/Storyboard_10.jpg}
        \label{fig:Storyboard_10}
    \end{figure}
    
    \newpage

    \begin{figure}[H]
        \vspace{+0.5cm}
        \hspace{+0.4cm}
        \includegraphics[angle= -90, width=0.95\textwidth]{Images/Storyboard_11.jpg}
        \label{fig:Storyboard_11}
    \end{figure}

    \newpage

    \begin{figure}[H]
        \vspace{-0.63cm}
        \hspace{-1.8cm}
        \includegraphics[angle= -90, width=1.05\textwidth]{Images/Storyboard_12.jpg}
        \label{fig:Storyboard_12}
    \end{figure}
    
    \newpage

    \begin{figure}[H]
        \vspace{-0.63cm}
        \hspace{-1.8cm}
        \includegraphics[angle= -90, width=1.05\textwidth]{Images/Storyboard_13.jpg}
        \label{fig:Storyboard_13}
    \end{figure}
    
    \newpage

    \begin{figure}[H]
        \vspace{-0.63cm}
        \hspace{-1.8cm}
        \includegraphics[angle= -90, width=1.05\textwidth]{Images/Storyboard_14.jpg}
        \label{fig:Storyboard_14}
    \end{figure}

    \newpage

    \begin{figure}[H]
        \vspace{-0.63cm}
        \hspace{-1.8cm}
        \includegraphics[angle= -90,width=1.05\textwidth]{Images/Storyboard_15.jpg}
        \label{fig:Storyboard_15}
    \end{figure}


\subsection{Τεχνικές που εφαρμόστηκαν}
    Στην απεικόνιση των σκηνών μας προσέξαμε να υπακούμε σε διάφορες τεχνικές:
    \begin{itemize}
        \item Όταν τοποθετούμε την πρωταγωνίστρια σε ένα διαφορετικό περιβάλλον φροντίζουμε να τραβήξουμε πρώτα ένα Establishing Long Shot το οποίο έχει ως στόχο να μπορέσει ο θεατής το τοποθετηθεί και ο ίδιος στον χώρο.
        \item Φροντίζουμε να υπακούμε στον κανόνα των τρίτων ή τον Χρυσό κανόνα όπου μπορούμε.
        \item Εφαρμόζουμε POV shots σε στιγμές που μας ενδιαφέρει να δούμε τον κόσμο από τα μάτια της πρωταγωνίστριας
        \item Κάνουμε κοντινά στα πρόσωπα των χαρακτήρων για να δείξουμε τα συναισθήματα αυτών πιο καθαρά
        \item Σε συνδυασμό με το POV χρησιμοποιούμε Over the shoulder shots για να προβάλουμε την κατεύθυνση της οπτικής γωνίας του εκάστοτε χαρακτήρα μας.
    \end{itemize}

    Πολύ χρήσιμες αποτέλεσαν για την πιστή αποτύπωση των σκηνών μας "φωτογραφίες-αναφορά" που χρησιμοποιήσαμε για να αποτυπώσουμε επακριβώς την διάταξη των αντικειμένων εντός του πλάνου.
    Παραδείγματα:
    \begin{figure}

    \end{figure}

    Η μετ' έπειτα ψηφιοποίση τους έγινε μέσω Camscanner και επεξεργασία σε Photoshop για να μην φαίνεται το λογότυπο.

\subsection{Προσαρμοστικές τροποποιήσεις}

\subsubsection{Χρόνος}
    Η αποτύπωση της διάταξης των σκηνών μας στο Storyboard αποδείχθηκε μια αρκετά χρονοβόρα διαδικασία. Οι σκηνές στις οποίες καταλήξαμε στο συγκεκριμένο στάδιο της παραγωγής ήταν 85. Τα σκίτσα πραγματοποιήθηκαν στο χέρι αξιοποιώντας ένα template που επέτρεπε ταυτόχρονο σχολιασμό και αποτύπωση 6 πλάνων ανά σελίδα. Υπολογίζοντας περίπου 15 λεπτά η μία ο τελικός χρόνος ολοκλήρωσης ανήλθε στα 1275 λεπτά ή 21.25 ώρες που μοιράστηκαν σε 6 μέρες με κατά προσέγγιση 4 ώρες ανά φορά.

\newpage

\section{Κάμερα}

    Αποφασίσαμε να τραβήξουμε την ταινία μας στα 50fps το οποίο σημαίνει ότι εναλλασσόμασταν μεταξύ shutter speed 50 και 100 σε όλες μας τις λήψεις, ανάλογα με τις συνθήκες φωτισμού. Η ανάγκη επιμονής σε αυτές τις δύο τιμές shutter speed προέρχεται από το γενονός ότι...(Δημητρη δωσε μας τα φώτα σου)
    \vspace{0.5cm}
    Στις πρωινές σκηνές είχαμε την ελευθερία να μεταβάλλουμε τις τιμές shutter speed, aperature και gain για να αποτυπώσουμε ακριβώς αυτό που θέλουμε. Η ιεράρχηση των επιλογών αυτών ακολουθούν την ακόλουθη δομή. Επιλέγουμε να κρατήσουμε το ISO όσο πιο χαμηλό γίνεται για να έχουμε τον λιγότερο δυνατό θόρυβο. Αυτή η ρύθμιση δεν ενισχύει το σήμα μας οπότε οι συνθήκες φωτισμού πρέπει να είναι αρκετά φωτεινές έτσι ώστε να μας επιτραπεί η ελαχιστοποίηση του ISO. Δευτερεύον, όμως επίσης σημαντικό αποτελεί το κλείστρο. Συνήθως επιλέγουμε χαμηλή τιμή επιτρέποντας περισσότερο φως να εισέλθει στον αισθητήρα και αυξάνοντας παράλληλα το βάθος πεδίου. Η τελευταία παράμετρος της κάμερα είναι το shutter speed το οποίο εναλλάσσαμε σε 50 ή 100 fps ανάλογα με τις συνθήκες φωτισμού. Μικρο shutter speed σημαίνει περισσότερη έκθεση σε φως ανά frame το οποίο φωτίζει το πλάνο μας. Υψηλό shutter speed έχει το ακριβώς αντίθετο αποτέλεσμα.
    
    \vspace{0.5cm}  
    
    Παραδείγματα \\
    
    Όταν κάναμε λήψεις σε εξωτερικό χώρο πρωί-μεσημέρι αυξήσαμε την τιμή του κλείστρου επειδή είχαμε Long shots όπου δεν χρειαζόμαστε βάθος πεδίου και το τελικό αποτέλεσμα κατέληγε καμμένο, ακόμα και μετά 
    
    \vspace{0.5cm}
    
    Στις βραδινές μας λήψεις η αδυναμία μας να χρησιμοποιήσουμε φώτα σε δημόσιους χώρους και τα ανεπαρκή χαρακτηριστικά της κάμερα που χρησιμοποιήσαμε (Nikon D3300, Lens 3.5-5.6 ελάχιστη τιμή κλείστρου, γωνίες θέασης 18-55mm) μας ανάγκασε να αυξήσουμε το gain (ISO) της κάμερας στα 3200-6400, ενισχύοντας τόσο το σήμα όσο και τον θόρυβο που διέπει αυτό. Παράλληλα ελαχιστοποιήσαμε την τιμή του κλείστρου και του shutter speed. Ως εκ τούτου, για να αποφύγουμε τον θόρυβο, προσθέσαμε, στο post-processing στάδιο, blur effects τα οποία συνάμα δίνουν και την αίσθηση της μέθης.
    
    \vspace{0.5cm}
    
    Στις λήψεις του μετρό δεν μπορέσαμε να αποτυπώσουμε επ' ακριβώς το σκηνοθετικό μας όραμα καθώς οι φύλακες μας έδωσαν περιορισμένο χρόνο λήψεων και έπρεπε να αποφύγουμε την προβολή κόσμου εντός των πλάνων μας. Έτσι, ακόμα και μετά επιλογή απόμακρων σταθμών και αδρανών ωρών η συνεχής μεταβολή των συνθηκών λόγω εισερχόμενου κόσμου μας ανάγκασε να πλησιάσουμε περισσότερο στους ηθοποιούς και να τους τραβήξουμε υπό γωνίες που δεν φαίνεται κανένας άλλος επιβάτης.

\section{Φώτα}

\section{Ηχοληψία}

\section{Green Screen}

\section{Scheduling-Casting-People Management}

\section{Μοντάζ}

\section{To-Do List}
\begin{itemize}
    \item Αλλαγή τηλεκοντρολ στο σεναριο
    \item Αιτιολόγηση επιλογών παραγωγής
    \item Δυσκολίες και λύσεις αυτών
    \item Αναλυτικά η διδικασία των παραδοτέων
    \item Αφίσα
\end{itemize}

    

\end{document}